\documentclass[a4paper,12pt]{article} % добавить leqno в [] для нумерации слева

%%% Работа с русским языком
\usepackage[margin=0.5in]{geometry}
\usepackage{cmap}					% поиск в PDF
\usepackage{mathtext} 				% русские буквы в формулах
\usepackage[T2A]{fontenc}			% кодировка
\usepackage[utf8]{inputenc}			% кодировка исходного текста
\usepackage[english,russian]{babel}	% локализация и переносы

%%% Дополнительная работа с математикой
\usepackage{amsmath,amsfonts,amssymb,amsthm,mathtools} % AMS
\usepackage{icomma} % "Умная" запятая: $0,2$ --- число, $0, 2$ --- перечисление

%% Номера формул
\mathtoolsset{showonlyrefs=true} % Показывать номера только у тех формул, на которые есть \eqref{} в тексте.

%% Шрифты
\usepackage{euscript}
\usepackage{mathrsfs} 


%% Перенос знаков в формулах (по Львовскому)
\newcommand*{\hm}[1]{#1\nobreak\discretionary{}
{\hbox{$\mathsurround=0pt #1$}}{}}

%%% Заголовок
\author{}
\title{Домашнее задание по программированию}
\date{}

\begin{document}
	\maketitle
	\begin{enumerate}
		\item Определить интерфейс IMetric, членами которого являются два свойства типа double доступных только для чтения: Mass - масса объекта и Distance - расстояние от объекта до начала координат.
		
		Определить классы точек в двумерном (Point2D) и трехмерном (Point3D) пространствах. Класс Point2D должен содержать два свойства X и Y - координаты на плоскости, а класс Point3D - свойства X, Y и Z типа double.
		
		Реализовать интерфейс IMetric в этих классах. Mass возвращает всегда 0, а Distance - расстояние от объекта до начала координат, т.е. $\sqrt{X^2+Y^2}$ или $\sqrt{X^2+Y^2+Z^2}$.
		
		Определить обобщенный класс MaterialPoint<T>, объект которого представляет материальную точку, т.е. точку, обладающую массой. При этом на T надожить ограничения - реализацию интерфейса IMetric. В классе определить свойство Coordinates типа T - координаты центра материальноый точки
		
		Реализовать интерфейс IMetric в классе MaterialPoint<T>. Mass - возвращает массу точки, а Distance - расстояние от центра объекта до начала координат.
		
		Определить обобщенный класс CloudOfPoints<T>, представляющий собой "облако"\ точек. На тип T наложить ограничение - реализацию интерфейся IMetric. Список точек, формирующих "облако"\ должен храниться в обобщенном списке List<T>.
		Определить метод Add(T item), который добавляет в список точек новую точек. Определить пустой конструктор и конструктор с params. 
		
		
		Определить в классе CloudOfPoints<T> свойства: double TotalMass - сумма масс всех точек, double Radius - расстояние от начала координат до самой дальней точки "облака".
		
		В классе CloudOfPoints<T> переопределить метод ToString(), чтобы он возвращал общее количество точек в "облаке"\, массу облака и его радиус в формате:
		$$\text{ Всего точек: <N>. Масса: <M>. Радиус <R>.}$$
		
		Реализовать в классе CloudOfPoints<T> интерфейс IComparable. Большим считать облако с большей массой.
		
		В основной программе создать по 2 объекта типа CloudOfPoints, с классами MaterialPoints<Point2D> и MaterialPoints<Point3D>. В каждый из них добавить 10 точек с случайными целыми массами в диапазоне от [1,20) и целыми координатами в диапазоне от [0,100). Сравнить и вывести 2 самых больших объекта.
	\end{enumerate}
\end{document}