\documentclass[a4paper,12pt]{article} % добавить leqno в [] для нумерации слева

%%% Работа с русским языком
\usepackage[margin=0.5in]{geometry}
\usepackage{cmap}					% поиск в PDF
\usepackage{mathtext} 				% русские буквы в формулах
\usepackage[T2A]{fontenc}			% кодировка
\usepackage[utf8]{inputenc}			% кодировка исходного текста
\usepackage[english,russian]{babel}	% локализация и переносы

%%% Дополнительная работа с математикой
\usepackage{amsmath,amsfonts,amssymb,amsthm,mathtools} % AMS
\usepackage{icomma} % "Умная" запятая: $0,2$ --- число, $0, 2$ --- перечисление

%% Номера формул
\mathtoolsset{showonlyrefs=true} % Показывать номера только у тех формул, на которые есть \eqref{} в тексте.

%% Шрифты
\usepackage{euscript}
\usepackage{mathrsfs} 


%% Перенос знаков в формулах (по Львовскому)
\newcommand*{\hm}[1]{#1\nobreak\discretionary{}
{\hbox{$\mathsurround=0pt #1$}}{}}

%%% Заголовок
\author{}
\title{Домашнее задание по программированию}
\date{}
\begin{document}
	\maketitle
	
	\begin{enumerate}
		\item \noindent В библиотеке классов Animal определить класс Animal со следующими свойствами, имеющими только get-акцессор:
		\begin{enumerate}
			\item[a] Строка Name с именем (кличкой) животных.
			\item[b] Целочисленное поле year, содержащее возраст животного.
		\end{enumerate}
		Создать класс Mammal, наследуемый от Animal. В классе Mammal определить:
		\begin{enumerate}
			\item[a] Целочисленное поле milkMax, которое содержит информацию, сколько максимально литров молока животное даёт в неделю.
			\item[b] Метод GetMilk(), который возвращает число, сколько молока животное дало сейчас по формуле
			\begin{equation*}
			GetMilk() = 
			\begin{cases}
			0, \text{если } year \in (0;3) 
			\\
			milkMax +3 - year,  \text{если }2 < year < milkMax,
			\\
			0, year> milkMax
			\end{cases}
			\end{equation*}
			
		\end{enumerate}
	Создать класс Cow, наследуемый от Mammal. В классе Cow определить метод void Voice(), который выводит на консоль сообщения типа "Я коровка му-му-му, я даю n литров молока". 
	
	Где n - количество молока, которое она даёт.
	
	Создать класс Goat, наследуемый от Mammal. В классе Goat определить метод void Voice(), который выводит на консоль сообщения типа "Я коза бе-бе-бе, я даю n литров молока". 
	
	Где n - количество молока, которое она даёт.
	
	В основной программе создать массив string[] names с кличками животных (заранее записанных в файле Names.txt в строку, разделенные пробелами (файл находится в каталоге решения)). Затем создать массив Animal[] animalShop, в котором с вероятностью 45\% записывать козу и с вероятностью 55\% записывать корову. Имя - случайное из заранее заданного списка. Количество лет и максимальное количество молока - случайное целое число из диапазона [1, 15) для коровы и [1,7) для козы. Для каждого объекта после создания вызвать метод Voice().
	
	С помощью одного оператора foreach посчитать суммарное количество молока, которые дают козы и коровы, и вывести в файл output.txt "Cows"\, если молока дают больше коровы, и "Goats"\ - если козы.
	
	
	\end{enumerate}
	
\end{document}