\documentclass[a4paper,12pt]{article} % добавить leqno в [] для нумерации слева

%%% Работа с русским языком
\usepackage[margin=0.5in]{geometry}
\usepackage{cmap}					% поиск в PDF
\usepackage{mathtext} 				% русские буквы в формулах
\usepackage[T2A]{fontenc}			% кодировка
\usepackage[utf8]{inputenc}			% кодировка исходного текста
\usepackage[english,russian]{babel}	% локализация и переносы

%%% Дополнительная работа с математикой
\usepackage{amsmath,amsfonts,amssymb,amsthm,mathtools} % AMS
\usepackage{icomma} % "Умная" запятая: $0,2$ --- число, $0, 2$ --- перечисление

%% Номера формул
\mathtoolsset{showonlyrefs=true} % Показывать номера только у тех формул, на которые есть \eqref{} в тексте.

%% Шрифты
\usepackage{euscript}
\usepackage{mathrsfs} 


%% Перенос знаков в формулах (по Львовскому)
\newcommand*{\hm}[1]{#1\nobreak\discretionary{}
{\hbox{$\mathsurround=0pt #1$}}{}}

%%% Заголовок
\author{}
\title{Домашнее задание по программированию}
\date{}
\begin{document}
	\maketitle
	
	\begin{enumerate}
		\item Создайте программу, котороая записывает в файл "Fib.txt"\ в каталоге с решением первые n чисел Фибоначчи. Число n считывать в программе. Использовать файловые потоки. Обработать все возможные исключения.
		 
		 \item С помощью класса DriveInfo и его статических методов получте название и объем всех дисков на компьютере. Выведите информацию о них в файл "DriveInfo.txt". Используйте файловые потоки и обработайте все исключения.
		 
		 \item С класса DirectoryInfo получите список всех файлов в каталоге с решением вашей программы. Создайте файл в каталоге с решением "MyFiles.txt"\ и с помощью файловых потоков запишите в него все файлы, которые в своем названии содержат "pro"\ (регистр букв не учитывается). Обработайе все исключения.
		 
	\end{enumerate}
	
\end{document}