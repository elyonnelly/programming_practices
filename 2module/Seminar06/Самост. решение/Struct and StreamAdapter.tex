\documentclass[a4paper,12pt]{article} % добавить leqno в [] для нумерации слева

%%% Работа с русским языком
\usepackage[margin=0.5in]{geometry}
\usepackage{cmap}					% поиск в PDF
\usepackage{mathtext} 				% русские буквы в формулах
\usepackage[T2A]{fontenc}			% кодировка
\usepackage[utf8]{inputenc}			% кодировка исходного текста
\usepackage[english,russian]{babel}	% локализация и переносы

%%% Дополнительная работа с математикой
\usepackage{amsmath,amsfonts,amssymb,amsthm,mathtools} % AMS
\usepackage{icomma} % "Умная" запятая: $0,2$ --- число, $0, 2$ --- перечисление

%% Номера формул
\mathtoolsset{showonlyrefs=true} % Показывать номера только у тех формул, на которые есть \eqref{} в тексте.

%% Шрифты
\usepackage{euscript}
\usepackage{mathrsfs} 


%% Перенос знаков в формулах (по Львовскому)
\newcommand*{\hm}[1]{#1\nobreak\discretionary{}
{\hbox{$\mathsurround=0pt #1$}}{}}

%%% Заголовок
\author{}
\title{Домашнее задание по программированию}
\date{}
\begin{document}
	\maketitle
	
	\begin{enumerate}
		\item В библиотеке классов LibHSE создайте структуру Subject, в которой определены следующие свойства с get - аксессором:
		\begin{enumerate}
			\item string NmeOfSubject - название предмета
			\item int Mark - целочисленное число от 0 до 10
			
		\end{enumerate}
		Создайте структуру ExamBook со следующими полями и свойсвами
		\begin{enumerate}
			\item Конструктор, который принимает параметры Subject[] subjects и string name
			\item Name  - свойство с get-аксессором
			\item Subject[] subjects- массив предметов 
			\item свойство AverageMark - средняя оценка в зачетке
			\item свойство bool haveNegativeMarks - которое сообщает, есть ли в зачкетке оценки <4
			\item Метод string ToSave() - возвращает информацию о зачетке вида
			
			
		Egor: Math 7;
			 Programming 5;
			 History 10; 
			 
			Dima: English 3;
			 \
		\end{enumerate}
		 В первом проекте создать массив ExamBook[] examBooks из 10 элементов класса ExamBook. В каждом объекте объявить массив Subject[] subjects случайной длины из диапазона [1,10) 
		 Определить каждую оценку случайно (диапазон [1,10]), с названием предмета и именем студента (название предмета и имя генерировать случайным образом случайной длины).
		 
		 С помощью адаптеров потока запишите examBooks в файл "ListHSE.txt" (файл находится в каталоге с решением) информацию о всех зачетках (используйте метод ToSave())
		 
		 Во второй программе создайте массив из 10 элементов класса ExamBook[] examBooks. С помощью адаптера потока считывайте файл "ListHSE" (обработайте все возможные исключения), и запишите их в массив examBooks. Выведете на экран количество зачеток, у которых нет незачей, и их все средние оценки. Далее выведете количество зачеток, у которых есть хотя бы один незачет, и название незачтенного предмета (выведете первый попавшийся).
		 
		 
	\end{enumerate}
	
\end{document}