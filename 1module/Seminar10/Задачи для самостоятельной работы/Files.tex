\documentclass[a4paper,12pt]{article} % добавить leqno в [] для нумерации слева

%%% Работа с русским языком
\usepackage[margin=0.5in]{geometry}
\usepackage{cmap}					% поиск в PDF
\usepackage{mathtext} 				% русские буквы в формулах
\usepackage[T2A]{fontenc}			% кодировка
\usepackage[utf8]{inputenc}			% кодировка исходного текста
\usepackage[english,russian]{babel}	% локализация и переносы

%%% Дополнительная работа с математикой
\usepackage{amsmath,amsfonts,amssymb,amsthm,mathtools} % AMS
\usepackage{icomma} % "Умная" запятая: $0,2$ --- число, $0, 2$ --- перечисление

%% Номера формул
\mathtoolsset{showonlyrefs=true} % Показывать номера только у тех формул, на которые есть \eqref{} в тексте.

%% Шрифты
\usepackage{euscript}	 % Шрифт Евклид
\usepackage{mathrsfs} % Красивый матшрифт


%% Перенос знаков в формулах (по Львовскому)
\newcommand*{\hm}[1]{#1\nobreak\discretionary{}
{\hbox{$\mathsurround=0pt #1$}}{}}

%%% Заголовок
\author{}
\title{Домашнее задание по программированию}
\date{}
\begin{document}
	\maketitle
	\begin{enumerate}
		\item \noindent Написать программу, которая записывает в файл BIN.txt натуральное число в двоичной систнме счисления. При первом запуске программы создать файл BIN.txt и записать в него число "0". При последующих запусках, программа должна изменять число на один. Повтор решения обязателен. Необходимо обработать исключения, при которых программа может завершиться аварийно.
		
		Пример: после первого запуска файл должен содержать "0". После второго - "1". После третьего - "10". После четвертого - "11".
		
		\item  \noindent ТиПИчный студент долго писал своею курсовую работу. Он хочет узнать, сколько слов, и сколько предложений в его курсовой работе, а так же среднее количество слов в одном предложении (округление происходит по правилам математики). Файл формата txt положить в каталог с решением. Название файла - "MyWork.txt". Вывод должен происходить в консоли в формате:
		
		<Количество слов>
		
		<Количество предложений>
		
		<Количество слов / количество предложений> 
		
		Предложения разделяются только символом ".".
		
		\item Программа должна вывести на консоль свой код.
	\end{enumerate}
\end{document}